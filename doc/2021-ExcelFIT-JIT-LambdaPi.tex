\documentclass{ExcelAtFIT}

%\ExcelFinalCopy
\hypersetup{
	pdftitle={Paper Title},
	pdfauthor={Author},
	pdfkeywords={Keyword1, Keyword2, Keyword3}
}
\lstset{
	backgroundcolor=\color{white},   % choose the background color; you must add \usepackage{color} or \usepackage{xcolor}; should come as last argument
	basicstyle=\footnotesize\tt,        % the size of the fonts that are used for the code
}
\ExcelYear{2021}
\PaperTitle{How to Write an Excellent Excel@FIT Paper}
\Authors{Adam Herout*}
\affiliation{*%
  \href{mailto:herout@fit.vutbr.cz}{herout@fit.vutbr.cz},
  \textit{Faculty of Information Technology, Brno University of Technology}}

\Keywords{Keyword1 --- Keyword2 --- Keyword3}

\Supplementary{\href{http://youtu.be/S3msCdn3fNM}{Demonstration Video} --- \href{http://excel.fit.vutbr.cz/}{Downloadable Code}}


\Abstract{
What is the problem? What is the topic?, the aim of this paper?
\phony{Lorem ipsum dolor sit amet, consectetur adipiscing elit. Fusce ullamcorper suscipit euismod. Mauris sed lectus non massa molestie congue. In hac habitasse platea dictumst.}
%
How is the problem solved, the aim achieved (methodology)?
\phony{Lorem ipsum dolor sit amet, consectetur adipiscing elit. Fusce ullamcorper suscipit euismod. Mauris sed lectus non massa molestie congue. In hac habitasse platea dictumst. Curabitur massa neque, commodo posuere fringilla ut, cursus at dui. Nulla quis purus a justo pellentesque.}
%
What are the specific results? How well is the problem solved?
\phony{Lorem ipsum dolor sit amet, consectetur adipiscing elit. Fusce ullamcorper suscipit euismod. Mauris sed lectus non massa molestie congue. In hac habitasse platea dictumst.}
%
So what? How useful is this to Science and to the reader?
\phony{Lorem ipsum dolor sit amet, consectetur adipiscing elit. Fusce ullamcorper suscipit euismod.}
}

\Teaser{
	\TeaserImage{placeholder.pdf}
	\TeaserImage{placeholder.pdf}
	\TeaserImage{placeholder.pdf}
}

\begin{document}

\startdocument


\section{Introduction}

\textbf{[Motivation]} What is the raison d'\^{e}tre of your project? Why should anyone care? No general meaningless claims. Make bulletproof arguments for the importance of your work.

\textbf{[Problem definition]} What exactly are you solving? What is the core and what is a bonus? What parameters should a proper solution of the problem have? Define the problem precisely and state how its solution should be evaluated.

\textbf{[Existing solutions]} Discuss existing solutions, be fair in identifying their strengths and weaknesses. Cite important works from the field of your topic. Try to define well what is the \textit{state of the art}. You can include a Section 2 titled ``Background'' or ``Previous Works'' and have the details there and make this paragraph short. Or, you can enlarge this paragraph to a whole page. In many scientific papers, \emph{this} is the most valuable part if it is written properly.

\textbf{[Our solution]} Make a quick outline of your approach -- pitch your solution.  The solution will be described later in detail, but give the reader a very quick overview now.

\textbf{[Contributions]} Sell your solution. Pinpoint your achievements. Be fair and objective.


The aim of this work is to create an efficient compiler for a language with
dependent types. Compiling dependently-typed languages, compared to simply-typed
or untyped languages, comes with large performance penalties, often making large
non-trivial programs hard to write due to the compiler running out of memory or
taking minutes to compile.
https://github.com/agda/agda/issues/514#issuecomment-129023737
https://github.com/idris-lang/Idris2/issues/964
On the other hand, programming language research projects place more and more
demands on the compiler  (cubical, univalence, ... CITE)

The goal of this project is to create a compiler for a dependently-typed
language that outperforms existing compilers at, in particular, the speed of
type-checking and compilation; evaluation speed of the resulting program is not
as important for our benchmarks. Special attention will be on the asymptotics,
as the implementation platform brings rather large constant factors.

This work is a successor to the Cadenza project (cite) which implements a
simply-typed lambda calculus with extensions in the Truffle framework. While it
is unfinished and did not show promising performance compared to other
simply-typed lambda calculus implementations according to its creator Edward
Kmett, my work attempts to apply its ideas to the dependently-typed lambda
calculus, where the presence of type-level computation should lead to larger
gains.

I have not found another attempt to apply just-in-time compilation in the
implementation of a language with dependent types and in type-level
computations in particular. I assume that it is due to the fact that the Truffle
project is not well-known and implementing a just-in-time compiler without it is
not straightforward.

--- Using the technique of just-in-time compilation we can avoid the performance
problems associated with dependent type-checking by rewriting inefficient parts
of the control-flow graph during compilation.

--- The Truffle language implementation framework gives us a way of turning an
interpreter into a compiler with few changes, and allows us to rewrite the
control-flow graph and just-in-time compilation process during runtime.

--- This is a novel approach to a problem all current dependently-typed
languages face and, if successful, will bring immediate benefits to programming
language implementers.


\section{How To Use This Template}
\label{sec:HowToUse}

Here will go several sections describing \textbf{your work}. From theoretical background (Section 2), through your own methodology (Section 3), experiments and implementation (Section 4 and possibly 5), to conclusions (Section 6). Instead of such technical content, here in this template we give a few hints how to write the paper.

\section{How To Write the Paper --- A Few Hints}
\label{sec:HowToWrite}

A reasonable way to start writing is sketching the \textbf{abstract} \cite{Herout-Abstract}.  Writing the abstract helps focus on what is important in the paper, what is the contribution, the meaning for the community.  This exercise might take some 20 minutes and it pays back by clearing the key points of the text.
In 99\,\% cases it is very reasonable to stick to the abstract structure \cite{Lebrun2011} which is provided in this template.

Once you have the abstract, it should be very clear what is the message of the paper, what is the newly introduced knowledge, what are the proofs of its contribution, etc.  This is the right time to start constructing the \emph{skeleton} of the paper: its \textbf{comics edition}~\cite{Herout-Comics}.
This thing is composed of mainly four items:
\begin{enumerate} [noitemsep]
	\item \textbf{Sections and subsections.}
	\item \textbf{Figures and tables.}  At this phase, knowing that ``once there will be a figure about this and that'' is just fine.  That is why we have the \textit{placeholder.pdf} image -- see Figure~\ref{fig:WidePicture}.  If this totally generic image can be replaced by some temporary image which still needs more work, but which is closer to the target version, go ahead. A hand-drawing photographed by a cellphone is perfect at this stage.
	\item \textbf{Todo's.} In the early comics version, every section is filled by one or more \texttt{$\backslash$todo} commands and nothing else.  A todo in the text might look like: \todo{you should do something}.  Unlike some elaborated todo packages, this simple solution (defined in the template) does not break the page formatting and it is perfectly sufficient.
	\item \textbf{Phony placeholder texts.}  These help you estimate the proportions of individual sections and subsections and to better aim at the correct paper length. Use \texttt{$\backslash$blind\{3\}} to get three paragraphs of beautiful \phony{grey phony text}.
\end{enumerate}
One hour is usually enough for creating a nice comics edition of the paper.  No reason to wait, make a copy of the template and start butchering it.

Having the comics edition usually lubricates the whole writing process.  Now,
the paper contains 20 or so todos -- why not take the easiest one of them and
replace it with a few lines of text within 15 minutes or even less.  Writing is
no more a scary complex work.

\blind{1}

\section{Conclusions}
\label{sec:Conclusions}

\textbf{[Paper Summary]} What was the paper about, then? What the reader needs to remember about it?
\phony{Lorem ipsum dolor sit amet, consectetur adipiscing elit. Proin vitae aliquet metus. Sed pharetra vehicula sem ut varius. Aliquam molestie nulla et mauris suscipit, ut commodo nunc mollis.}

\textbf{[Highlights of Results]} Exact numbers. Remind the reader that the paper matters.
\phony{Lorem ipsum dolor sit amet, consectetur adipiscing elit. Sed tempus fermentum ipsum at venenatis. Curabitur ultricies, mauris eu ullamcorper mattis, ligula purus dapibus mi, vel dapibus odio nulla et ex. Sed viverra cursus mattis. Suspendisse ornare semper condimentum. Interdum et malesuada fames ac ante ipsum.}

\textbf{[Paper Contributions]} What is the original contribution of this work? Two or three thoughts that one should definitely take home.
\phony{Lorem ipsum dolor sit amet, consectetur adipiscing elit. Praesent posuere mattis ante at imperdiet. Cras id tincidunt purus. Aliquam erat volutpat. Morbi non gravida nisi, non iaculis tortor. Quisque at fringilla neque.}

\textbf{[Future Work]} How can other researchers / developers make use of the results of this work?  Do you have further plans with this work? Or anybody else?
\phony{Lorem ipsum dolor sit amet, consectetur adipiscing elit. Suspendisse sollicitudin posuere massa, non convallis purus ultricies sit amet. Duis at nisl tincidunt, maximus risus a, aliquet massa. Vestibulum libero odio, condimentum ut ex non, eleifend.}

\phantomsection
\bibliographystyle{unsrt}
\bibliography{bibliography}
\end{document}