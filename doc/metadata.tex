\projectinfo{
  project={DP},
  year={2021},
  date=\today,
  title.cs={Just-in-time překlad závisle typovaného lambda kalkulu},
  title.en={Just-in-Time Compilation\\of the Dependently-Typed Lambda Calculus},
  title.length={15.5cm},
  %sectitle.length={14.5cm}, % nastavení délky bloku s druhým titulkem pro úpravu zalomení řádku (lze definovat zde nebo níže)
  author.name={Jakub},
  author.surname={Zárybnický},
  author.title.p={Bc.},
  department={UITS},
  supervisor.name={Ondřej},
  supervisor.surname={Lengál},
  supervisor.title.p={Ing.},
  supervisor.title.a={Ph.D.},
  faculty={FIT},
  faculty.cs={Fakulta informačních technologií},
  faculty.en={Faculty of Information Technology},
  department.cs={Ústav inteligentních systémů},
  department.en={Department of Intelligent Systems},
  keywords.cs={Truffle, Virtuální stroj JVM, just-in-time překlad, tvorba překladačů, závislé typy, lambda kalkul},
  keywords.en={Truffle, Java Virtual Machine, just-in-time compilation, compiler construction, dependent types, lambda calculus},
  abstract.en={
   A number of programming languages have managed to greatly improve their
   performance by replacing their custom runtime system with general platforms
   that use just-in-time optimizing compilers like GraalVM or RPython. This
   thesis evaluates whether such a transition would also benefit
   dependently-typed programming languages or theorem provers.

   This thesis introduces the type-theoretic notion of dependent types and the
   algorithms involved in working with them, specifies a minimal
   dependently-typed language on the $\lambda\Pi\text{-calculus}$, and presents
   the implementation two interpreters for this language: a simple interpreter
   written in Kotlin, and a second interpreter, also written in Kotlin, that
   uses the Truffle language implementation framework on the GraalVM platform,
   which is a partial evaluation-based just-in-time compiler based on the Java
   Virtual Machine. The performance of these two interpreters is then compared
   on a number of normalization and elaboration tasks.

   [...specific numbers] abstract - primarily for evaluation, not too suitable for elab
  }, abstract.cs={
Řada programovacích jazyků byla schopna zvýšit svoji rychlost výměnou běhových
systémů stavěných na míru za obecné platformy, které pro optimalizaci
používají just-in-time překlad, jako jsou GraalVM nebo RPython. V této práci
vyhodnocuji, zda je použití takovýchto platforem vhodné i pro jazyky se
závislymi typy nebo důkazovými systémy.

Tato práce představuje koncepty $\lambda\text{-kalkulu}$ a teorie typů potřebné
pro úvod do závislých typů s relevantními algoritmy, specifikuje malý
závisle-typovaný jazyk založený na $\lambda\Pi\text{-kalkulu}$, a prezentuje dva
interpretery tohoto jazyka. Tyto interpretery jsou psané v jazyce Kotlin, první
je jednoduchý, psaný ve funkcionálním stylu a druhý používá platformu GraalVM a
Truffle. GraalVM je platforma založená na virtuálním stroji Javy (JVM), která
přidává just-in-time překladač založený na částečném vyhodnocení
(\textit{partial evaluation}) a Truffle je knihovna pro tvorbu programovacích
jazyků využívající tento překladač. Závěr práce vyhodnocuje běhové
charakteristiky těchto interpreterů na různých zátěžových testech.

   [...vyhodnocení]
  },
  extendedabstract={
  Systémy, které používají závislé typy, umožňují programátorům vytvářet
  programy, které jsou zaručeně správné vzhledem k vlastnostem, které mají
  uložené v typech. Tyto systémy je také možné použít pro logické nebo
  matematické důkazy, nebo pro dokazování správnosti celých systémů.  Poslední
  roky přinesly mnoho pokroků v teorii typů, na níž jsou tyto systémy založené,
  jako např. kvantitativní nebo homotopické typy. Vysoké nároky na důkazové schopnosti
  systémů s sebou ale prináší problémy s výkonem, konkrétně rychlostí kontroly
  typů (\textit{type-checking}, \textit{elaboration}). V této práci zhodnocuji
  vhodnost just-in-time překladu pro takové systémy, což je jeden z obecných
  přístupů pro optimalizaci rychlosti systémů.

  V první části vysvětluji principy typovaného $\lambda\text{-kalkulu}$, na němž
  jsou systémy se závislými typy založené, základy teorie typů a konkrétní
  specifikace několika rozšíření, které poté používám pro specifikaci malého
  jazyka.

  V druhé sekci pokračuji představením algoritmů, které jsou potřeba pro práci
  se závislými typy: \textit{normalization-by-evaluation} a
  \textit{bidirectional typing}. Tyto implementuji a používám pro vytvoření
  funkčního interpreteru tohoto jazyka.
REPL, full programming language, ...

  Třetí sekce představuje detaily platformy GraalVM a knihovny Truffle, které
  poté používám pro implementaci druhého interpreteru, který využívá
  just-in-time překladu.

  V závěru práce vyhodnocuji [...]

  Vyhodnocení, přínos práce
  },
  declaration={
    I hereby declare that this Master's thesis was created as an original work
    by the author under the supervision of Ing. Ondřej Lengál Ph.D.

    I have listed all the literary sources, publications, and other sources
    that were used during the preparation of this thesis.
  },
  acknowledgment={
    I would like to thank my supervisor Ing. Ondřej Lengál Ph.D. for entertaining a
    second one of my crazy thesis proposals.

    I would like to thank Edward Kmett for suggesting this idea: it has been a
    challenge.

    I would also like to thank my family and friends who supported me when I needed
    it the most throughout my studies.
  }
}
